\documentclass[a4paper,10pt]{article}
%\documentclass[a4paper,10pt]{scrartcl}

%\usepackage[utf8]{inputenc}
\usepackage[latin1]{inputenc}
\usepackage[spanish]{babel}
\title{}
\author{}
\date{}

\pdfinfo{%
  /Title    ()
  /Author   ()
  /Creator  ()
  /Producer ()
  /Subject  ()
  /Keywords ()
}
\title{Breve Tutorial sobre c�mo usar JNI}
\author{Ing. Lamberto Maza Casas}
\date{\today}
\begin{document}
\maketitle
%\maketitle
\section{Crear el c�digo Java}
\noindent Primero escribimos la clase java EjemploString
\begin{verbatim}
package u4.jni00;

public class EjemploString {

    public native String replaceString(String sourceString,
            String strToReplace, String replaceString);
    
    static {
        System.loadLibrary("BibliotecaString");
    }
    
    public static void main(String[] args) {
        EjemploString ex = new EjemploString();
        String str1 = "";
        String str2 = "";
        str1 = "Sky Black";
        str2 = ex.replaceString(str1, "Black", "Blue");
        System.out.println("La cadena antes: " + str1);
        System.out.println("La cadena despues: " + str2);
    }
}//end class EjemploString
\end{verbatim}
\section{Crear el c�digo y las bibliotecas nativas}
Para escribir el c�digo en C++, debemos utilizar la herramienta 
javah (in\-clui\-da en el JDK) para generar un archivo de cabecera. 
Este archivo de cabecera contiene los prototipos de las funciones 
que deben implementarse en C++. En primer lugar se compila el 
c�digo java y, a continuaci�n se ejecuta esta he\-rra\-mien\-ta 
con el archivo class. Opcionalmente para compilar la clase 
EjemploString podemos utilizar la herramienta ant con un archivo 
build.xml como el siguiente:
\begin{verbatim}
<project name="JNI_EjemploString" basedir="." default="main">

    <property name="src.dir"     value="newpkgroot"/>

    <property name="build.dir"   value="Build"/>
    <property name="classes.dir" value="${build.dir}/classes"/>
    <property name="imagenes.dir" value="${classes.dir}/Imagenes"/>
    <property name="jar.dir"     value="${build.dir}/jar"/>

    <property name="main-class"  value="u4.jni00.EjemploString"/>

    <target name="clean">
        <delete dir="${build.dir}"/>
    </target>

    <target name="compile">
        <mkdir dir="${classes.dir}"/>
        <javac srcdir="${src.dir}" destdir="${classes.dir}"/>
<!--
	<copy todir="${imagenes.dir}">
		<fileset dir="${src.dir}/images"/>
	</copy>
-->
    </target>

    <target name="jar" depends="compile">
        <mkdir dir="${jar.dir}"/>	
<jar destfile="${jar.dir}/${ant.project.name}.jar" basedir="${classes.dir}">
            <manifest>
                <attribute name="Main-Class" value="${main-class}"/>
            </manifest>
        </jar>
    </target>

    <target name="run" depends="jar">
        <java jar="${jar.dir}/${ant.project.name}.jar" fork="true"/>
<!--
cd Build/classes/
java -Djava.library.path=~/BibliotecaString/   u4.jni00.EjemploString
En el directorio ~/BibliotecaString/ debe estar el archivo 
libBibliotecaString.so
-->
    </target>

    <target name="clean-build" depends="clean,jar"/>

    <target name="main" depends="clean,run"/>

</project>
\end{verbatim}
Las ubicaciones de los archivos build.xml y EjemploString.java antes 
de compilar usando ant deben ser las siguientes:
\begin{verbatim}
.
|-- BibliotecaString
|-- build.xml
`-- newpkgroot
    `-- u4
        `-- jni00
            `-- EjemploString.java
\end{verbatim}
\eject
Despu�s de ejecutar el comando ant compile debemos tener algo como 
esto:
\begin{verbatim}
.
|-- BibliotecaString
|-- Build
|   `-- classes
|       `-- u4
|           `-- jni00
|               `-- EjemploString.class
|-- build.xml
|-- newpkgroot
    `-- u4
        `-- jni00
            `-- EjemploString.java
\end{verbatim}

Al ejecutar javah debemos especificar el 
nombre de la clase (no el nombre del archivo) como primer 
par�metro.
Despues necesitamos crear la biblioteca BibliotecaString: 
Si trabajamos en linux deberemos crear una biblioteca 
din�mica que deber� llamarse libBibliotecaString.so, y si 
trabajamos en Windows deberemos crear una biblioteca de 
enlace din�mico (DLL) que debera llamarse BibliotecaString.dll
\par
\noindent En linux podemos usar la herramienta make con el
siguiente archivo Makefile:
\begin{verbatim}
INCLUDE_DIRS:= ./BibliotecaString/include /usr/lib/jvm/java-6-openjdk/include/
SOURCE_DIRS:= ./BibliotecaString/src/
CFILES:= bibliotecastring.cpp
#ASM_FILES:= startup_ARMCM4.S
INCLUDE_FLAGS:=$(patsubst %,-I%,$(INCLUDE_DIRS))
#TCHIP=TM4C123GH6PM
#CC=arm-none-eabi-gcc
#CC=avr-gcc
#CC=gcc
CC=g++
CFLAGS=-g -Wall -fPIC

#ld -G BibliotecaString/objectfiles/*.o -o BibliotecaString/lib/libBibliotecaString.so -lc -lpthread
#ld: BibliotecaString/objectfiles/bibliotecastring.o: relocation R_X86_64_PC32 against symbol `_ZN7JNIEnv_17GetStringUTFCharsEP8_jstringPh' can not be used when making a shared object; recompile with -fPIC
#ld: final link failed: Bad value
#make: *** [libBibliotecaString.so] Error 1    (2015.07.02)

#CPPFLAGS= $(INCLUDE_FLAGS) -D__AVR_LIBC_DEPRECATED_ENABLE__
CPPFLAGS= $(INCLUDE_FLAGS)
vpath %.h $(INCLUDE_DIRS)
vpath %.cpp *.o $(SOURCE_DIRS)
COBJECTS:= $(patsubst %.cpp,%.o,$(CFILES))

all: libBibliotecaString.so
##libBibliotecaString.so: bibliotecastring.cpp
#libBibliotecaString.so: $(COBJECTS)
# gcc -c -I/usr/lib/jvm/java-6-openjdk/include/ MyCJavaInterface.c -o MyCJavaInterface.o
# gcc -c -I MyCJavaInterface.c -o MyCJavaInterface.o
# ld -G MyCJavaInterface.o -o libMyCJavaInterface.so -lm -lc -lpthread
$(COBJECTS): %.o: %.cpp
	mkdir -p BibliotecaString/objectfiles
	$(CC) $(CFLAGS) $(CPPFLAGS) -c $< -o BibliotecaString/objectfiles/$@

libBibliotecaString.so: $(COBJECTS)
	mkdir -p BibliotecaString/lib
	ld -G BibliotecaString/objectfiles/*.o -o BibliotecaString/lib/$@ -lc -lpthread 

.PHONY: clean rebuild
clean:
	rm -frv BibliotecaString/lib/
	rm -frv BibliotecaString/objectfiles/

rebuild: clean all

\end{verbatim}
En windows podemos usar por ejemplo el IDE DevC++ en la version 
5.11 (32 y 64 bits) para crear la biblioteca DLL. En el DevC++ se crea un 
proyecto DLL, se escribe el codigo de las funciones C/C++. y se compila 
para obtener la DLL. Puede que nos marque error(es) como que no encuentra 
algunoc archivos como jni.h. Hay que buscar en Tools (o Herramientas) las 
opciones del compilador para agregar directorios como
\begin{verbatim}
C:\Program Files\Java\jdk_8u45_algo\include 
\end{verbatim}

por �ltimo, para ejecutar el archivo .class de java desde la l�nea de 
comandos debemos hacer lo siguiente:
\begin{verbatim}
 cd Build/classes
 java -Djava.library.path=RutaANuestrabiblioteca pkgname.ClasePrincipal
\end{verbatim}
En linux RutaANuestrabiblioteca ser� la ruta al archivo 
libBibliotecaString.so, mientras que en windows, RutaANuestrabiblioteca 
ser� la ruta al archivo bibliotecaString.dll.\par

\noindent Por otra parte si usamos un IDE como el Netbeans, deberemos
\begin{verbatim}
Para ejecutar en el Netbeans de debe agregar en el menu
run
   \set project configuration
                             \customize

donde dice run, en VM option escribir
-Djava.library.path=rutaparala_biblioteca.dll

o bien en linux
-Djava.library.path=rutaparala_libbiblioteca.so
\end{verbatim}
Este HowTo esta hecho muy a prisa. Pero espero que le sea util a 
alguien.                (2015.07.02)



\end{document}
