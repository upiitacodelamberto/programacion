\documentclass[a4paper,10pt]{article}
%\documentclass[a4paper,10pt]{scrartcl}

%\usepackage[utf8]{inputenc}
\usepackage[latin1]{inputenc}

\title{}
\author{}
\date{}

\pdfinfo{%
  /Title    ()
  /Author   ()
  /Creator  ()
  /Producer ()
  /Subject  ()
  /Keywords ()
}

\begin{document}
%\maketitle
\section{Crear el c�digo Java}
\noindent Primero escribimos la clase java EjemploString
\begin{verbatim}
package u4.jni00;

public class EjemploString {

    public native String replaceString(String sourceString,
            String strToReplace, String replaceString);
    
    static {
        System.loadLibrary("BibliotecaString");
    }
    
    public static void main(String[] args) {
        EjemploString ex = new EjemploString();
        String str1 = "";
        String str2 = "";
        str1 = "Sky Black";
        str2 = ex.replaceString(str1, "Black", "Blue");
        System.out.println("La cadena antes: " + str1);
        System.out.println("La cadena despues: " + str2);
    }
}//end class EjemploString
\end{verbatim}
\section{Crear el c�digo y las bibliotecas nativas}
Para escribir el c�digo en C++, debemos utilizar la herramienta 
javah (in\-clui\-da en el JDK) para generar un archivo de cabecera. 
Este archivo de cabecera contiene los prototipos de las funciones 
que deben implementarse en C++. En primer lugar se compila el 
c�digo java y, a continuaci�n se ejecuta esta he\-rra\-mien\-ta 
con el archivo class. Opcionalmente para compilar la clase 
EjemploString podemos utilizar la herramienta ant con un archivo 
build.xml como el siguiente:
\begin{verbatim}
<project name="JNI_EjemploString" basedir="." default="main">

    <property name="src.dir"     value="newpkgroot"/>

    <property name="build.dir"   value="Build"/>
    <property name="classes.dir" value="${build.dir}/classes"/>
    <property name="imagenes.dir" value="${classes.dir}/Imagenes"/>
    <property name="jar.dir"     value="${build.dir}/jar"/>

    <property name="main-class"  value="u4.jni00.EjemploString"/>

    <target name="clean">
        <delete dir="${build.dir}"/>
    </target>

    <target name="compile">
        <mkdir dir="${classes.dir}"/>
        <javac srcdir="${src.dir}" destdir="${classes.dir}"/>
<!--
	<copy todir="${imagenes.dir}">
		<fileset dir="${src.dir}/images"/>
	</copy>
-->
    </target>

    <target name="jar" depends="compile">
        <mkdir dir="${jar.dir}"/>	
<jar destfile="${jar.dir}/${ant.project.name}.jar" basedir="${classes.dir}">
            <manifest>
                <attribute name="Main-Class" value="${main-class}"/>
            </manifest>
        </jar>
    </target>

    <target name="run" depends="jar">
        <java jar="${jar.dir}/${ant.project.name}.jar" fork="true"/>
<!--
cd Build/classes/
java -Djava.library.path=~/BibliotecaString/   u4.jni00.EjemploString
En el directorio ~/BibliotecaString/ debe estar el archivo 
libBibliotecaString.so
-->
    </target>

    <target name="clean-build" depends="clean,jar"/>

    <target name="main" depends="clean,run"/>

</project>
\end{verbatim}
Las ubicaciones de los archivos build.xml y EjemploString.java antes 
de compilar usando ant deben ser las siguientes:
\begin{verbatim}
.
|-- BibliotecaString
|-- build.xml
`-- newpkgroot
    `-- u4
        `-- jni00
            `-- EjemploString.java
\end{verbatim}
\eject
Despu�s de ejecutar el comando ant compile debemos tener algo como 
esto:
\begin{verbatim}
.
|-- BibliotecaString
|-- Build
|   `-- classes
|       `-- u4
|           `-- jni00
|               `-- EjemploString.class
|-- build.xml
|-- newpkgroot
    `-- u4
        `-- jni00
            `-- EjemploString.java
\end{verbatim}

Al ejecutar javah debemos especificar el 
nombre de la clase (no el nombre del archivo) como primer 
par�metro.
\begin{verbatim}
$ mkdir -p BibliotecaString/include/
$ cd Build/classes/
$ javah -d ../../BibliotecaString/include/ u4.jni00.EjemploString
\end{verbatim}
Despu�s de ejecutar estos comandos debemos tener
\begin{verbatim}
.
|-- BibliotecaString
|   `-- include
|       `-- u4_jni00_EjemploString.h
|-- Build
|   `-- classes
|       `-- u4
|           `-- jni00
|               `-- EjemploString.class
|-- build.xml
|-- newpkgroot
    `-- u4
        `-- jni00
            `-- EjemploString.java
\end{verbatim}
\noindent Ahora mostramos una posible implementaci�n de la biblioteca 
BibliotecaString: (libBibliotecaString.so en linux o 
BibliotecaString.dll en Windows).
\begin{verbatim}
/**
 * bibliotecastring.c Archivo de implementacion de la biblioteca
 * BibliotecaString
 */
#include "u4_jni00_EjemploString.h"

JNIEXPORT jstring JNICALL Java_u4_jni00_EjemploString_replaceString
  (JNIEnv *env, jobject obj, 
  jstring _srcString, jstring _strToReplace, jstring _replString){
  const char *searchStr, *findStr, *replStr, *found;
  jstring newStr=NULL;
  int index;

  searchStr=env->GetStringUTFChars(_srcString, NULL);
  findStr=env->GetStringUTFChars(_strToReplace, NULL);
  replStr=env->GetStringUTFChars(_replString, NULL);

  found=strstr(searchStr, findStr);

  if(found!=NULL){
    char *newStringTemp;
    index=found-searchStr;
    newStringTemp=new char[strlen(searchStr)+strlen(replStr)+1];
    strcpy(newStringTemp, searchStr);
    newStringTemp[index]=0;
    strcat(newStringTemp, replStr);
    strcat(newStringTemp, &searchStr[index+strlen(findStr)]);
    newString=env->NewStringUTF((const char*)newStringTemp);
  }
  env->ReleaseStringUTFChars(_srcString, searchStr);
  env->ReleaseStringUTFChars(_strToReplace, findStr);
  env->ReleaseStringUTFChars(_replString, replStr);

  return (newString);
} 
\end{verbatim}


\end{document}
