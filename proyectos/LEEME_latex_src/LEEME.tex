\documentclass[12pt]{article}
\textwidth=19.0cm%\textheight=10cm
\hoffset=-2.5cm
\begin{document}
\begin{flushright}
Martes 22 de septiembre de 2015
\end{flushright}
\noindent{\large\bf Encuadre del directorio {\tt proyectos} 
del repositorio}\\
\begin{center}
{\tt https://github.com/upiitacodelamberto/programacion}
\end{center}
En este directorio ({\tt programacion/proyectos}) encontrar\'as 
los siguientes directorios:
\begin{itemize}
\item {\tt simples}
\item {\tt workspace}
\end{itemize}
El contenido del directorio {\tt simples} son archivos java 
en los que se tienen programas cuyos c\'odigos fuente son 
solo unas cuantas decenas de l\'ineas. En este directorio 
podr\'as encontrar (entre otros) el programa {\tt Hola.java} 
que servir\'a como programa ``Hola Mundo'' para este curso.\par
En el directorio {\tt workspace} encontrar\'as los directorios 
correspondientes a proyectos java gestionados o administrados 
en alguna de dos formas que mencionar\'e m\'as adelante. En 
este momento el contenido del directorio {\tt workspace} es:
\begin{itemize}
\item {\tt gestionadoscon$\_$ant}
\end{itemize}
\noindent{\large\bf Gesti\'on de proyectos java usando {\tt ant}}\\
En el directorio {\tt gestionadoscon$\_$ant} planeo colocar al 
menos un proyecto java con el que te mostrar\'e c\'omo se puede 
hacer la gesti\'on de un proyecto usando \'unicamente la 
herramienta {\tt ant}.\par
\noindent{\large\bf Gesti\'on de proyectos java usando {\tt eclipse}}\\
Los dem\'as directorios que poco a poco estar\'e agregando 
al directorio {\tt workspace} ``normalmente'' corresponder\'an 
a proyectos gestionados a trav\'es del IDE Eclipsei (que como 
veremos m\'as tarde tambi\'en usa la herramienta {\tt ant}).
\vskip.1in
\noindent{\bf Glosario}
\begin{description}
\item[ant] Apache Ant: Herramienta del proyecto apache (https://projects.apache.org/project.html?ant)
\item[eclipse] Ambiente integrado de desarrollo para gesti\'on de proyectos de 
software con soporte para varios lenguajes de programaci\'on (incluido Java). 
Una ventaja de este programa es que ``no requiere instalaci\'on'' (en Windows o en Linux). 
Todo lo que hay que hacer para instalarlo es descomprimir el archivo 
{\tt .zip} o {\tt tar.gz} que se descarga de la p\'agina https://eclipse.org/downloads/
\item[IDE] Integrated Development Environment.
\end{description}
\end{document}
